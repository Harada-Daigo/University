\documentclass[twocolumn]{jarticle}
\usepackage[dvips]{graphicx}
\usepackage{ascmac}


\setlength{\columnsep}{5mm}
\setlength{\textwidth}{175mm}
\setlength{\textheight}{25cm}
\addtolength{\topmargin}{-25mm}
\addtolength{\oddsidemargin}{-7mm}
\addtolength{\evensidemargin}{-7mm}

\def\baselinestretch{.97}

\title{木}
\author{原田 大瑚} 
\date{2024/03/31}

\newtheorem{tem}{定理}
\newtheorem{lem}{補題}
\newtheorem{tei}{系}
\newtheorem{reidai}{例題}

\begin{document}
\maketitle
\section{木}
\textgt{木}とは連結な閉路を含まないグラフであり、閉路を持たないグラフを\textgt{森}という。森は連結でなくともよい。また、木である$G$の全域部分グラフを$G$の\textgt{全域木}という。\\

\subsection{親と子}
\indent 木のある隣接する2つ点の根に近い方を\textgt{親}、葉に近い方を\textgt{子}という。\\

\subsection{根付き木}
\indent 木のある点を\textgt{根}と指定した、木を\textgt{根付き木}という。根付き木の根以外の次数が$1$の点を\textgt{葉}といい、葉でも根でもない点を\textgt{内点}という。そして、葉以外の各点が2つ以下の\textgt{子}を持つ根付き木を\textgt{二分木}という。\\

\subsection{高さ}
\indent 木の\textgt{高さ}は根から最も経路の長さが長くなる点までの長さである。

\subsection{木の点と辺の数の関係}
\begin{tem}任意の木Tの辺の数は点の数より$1$だけ少ない。
すなわち、$|E(T)| = |V(T)| - 1$である。\end{tem}
$|V(T)|$に関する数学的帰納法で証明する。まず、点数が$1$の場合は辺は存在しないので、定理は成り立つ。\\
$T$を$|V(T)|=n(nは2より大きい)$である任意の木とし、点数が$n$未満の木に対しては定理が成り立つとする。系$1-1$から$T$には次数が$1$である点$v$が存在する。$T$から$v$と$v$に接続する
辺を取り除いたグラフ$T'$は明らかに連結で閉路を含まないので木である。よって仮定より、$$|E(T')|=|V(T')|-1$$である。また、$T'$は$T$から点と辺を$1$つづつ取り除いたグラフなので、
$$|V(T')|=|V(T)|-1$$
$$|E(T')|=|E(T)|-1$$である。よって、$$|E(T)|=|V(T)|-1$$である。
\subsection{高さと二分木の葉の数の関係}
\begin{tem}高さ$k$の2分木には$2^k$個以下の葉が存在する。\end{tem}
\indent 2分木$T$の高さ$h(T)$に関する数学的帰納法で証明する。まず、高さ$0$のである2分木は根がはでもあり、一つの葉が存在在するため定理は成り立つ。次に、$T$を点$r$を根とする$h(T)=k (kは1以上)$である任意の2分木とし、高さが$k$未満
の任意の2分木に対しては定理が成り立つと仮定する。$r$の次数は$1$または$2$であるのでそれぞれの場合について考える。\\
\indent (a)まず、$r$の次数が$1$である場合を考える。$r$の子を$r_1$とし、$T$から点$r$と、$r$と$r_1$をつなぐ辺を取り除いた、$r_1$を根とする$T$の部分木を$T_1$とする。$T_1$は、高さが$k$未満
の任意の2分木に対しては定理が成り立つという仮定から、$T_1$の葉の数は$2^{k-1}$以下である。$T$と$T_1$の葉の数は同じなので、$T$の葉の数も$2^{k-1}$以下である。\\
\indent (b)$r$の次数が$2$の場合も同様である。$r$の子をそれぞれ$r_1$、$r_2$とし、それぞれを根とする部分木の葉の数は(a)と同様に$2^{k-1}$以下である。よって、$T$の葉の数は$2^k$以下である。
\\以上より、定理が成り立つことがわかる。\\
\end{document}
